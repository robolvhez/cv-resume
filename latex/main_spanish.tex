% !TeX program = lualatex
\documentclass{simplecv}
\usepackage[utf8]{inputenc}
\usepackage[spanish]{babel}
\usepackage[margin=0.75in]{geometry}
\usepackage{hyperref}
\usepackage{dirtytalk}
\usepackage{multicol}
\usepackage[style=spanish]{csquotes}
\usepackage[backend=biber,style=authoryear,sorting=ynt]{biblatex}
\usepackage{lipsum}

\hypersetup{
    colorlinks = true,
    urlcolor   = AccentTeal
}

\addbibresource{mybiblio.bib}

\name{Ing. Roberto Olvera Hernández}
% \title{Analista de datos genómicos}
\title{Ingeniero de Procesos y Calidad}
\email{roberto.olvhez@gmail.com}
\location{Zapopan}{Jalisco}{MX}
\phone{+52}{33-1540-9988}
\linkedin{\href{https://www.linkedin.com/in/rob-olvhez/}{rob-olvhez}}
\github{\href{https://github.com/robolvhez}{robolvhez}}

\begin{document}
\setmainfont{XCharter} % Sans-serif. Body. Sans can also be set as main font. Good options include: {XCharter, Heuristica, QTBasker}
\setsansfont{Inter} % Sans. Titles. Good options include: {Inter, Roboto, Cabin}
\nocite{*}
% L = Añadir localizacion; vacio = no incluir
\makecvheader{L}
\thispagestyle{firstpage}

% ------------
% PERFIL PROFESIONAL:
{\small

}
% ------------
{\small\sffamily\setlength{\parskip}{6pt}%

}\vspace{0.75em}

% ------------
% EXPERIENCE
% ------------
\cvsection{\job Experiencia profesional y académica}

    \cvevent{Vendor Management Especialist}
    {GDL Development}
    {}
    {2024}
    {actual}
    {
        \begin{itemize}[leftmargin=*]
        \setlength{\itemsep}{0cm}
            \item Automatización de reportes de marcas.
        \end{itemize}
    }

    \cvevent{Ingeniero de Procesos y Automatización}
    {GDL Development}
    {}
    {2023}
    {2024}
    {
        \textbf{Logros:}
        \begin{itemize}[leftmargin=*]
        \setlength{\itemsep}{0cm}
            \item Automatización de reportes históricos de inventario usando scripts de R y SQL mediante conexiones ODBC/MySQL, eliminando la necesidad de generar múltiples archivos Excel y respaldando información accesible en una base de datos establecida.
            \item Automatización de análisis para inventario anual. Reducción de 3 días a 1 día con la implementación de un script de R.
            % \item Optimicé la productividad de etiquetado de piezas para consesión en un 120\%, liberando 5h para la preparación de embarques y disminuyendo la necesidad de colaboradores eventuales.
        \end{itemize}

        \textbf{Actividades:}
        \begin{itemize}[leftmargin=*]
        \setlength{\itemsep}{0cm}
            \item Mapeo de actividades en todos los departamentos de la empresa.
            \item Análisis de indicadores para el área de operaciones.
            \item Planeación de compra de insumos para el departamento web.
            %\item Elaboración de guías visuales para colaboradores en operaciones.
        \end{itemize}
    }

    \cvevent{Trainee de Diagnóstico en Salud Animal}
    {Boehringer Ingelheim --- Vetmedica}
    {}
    {2021}
    {2022}
    {%
        \vspace{-11pt}
        \begin{itemize}[leftmargin=*]
        \setlength{\itemsep}{0cm}
            %\item Protocolos de Next Generation Sequencing (NGS) para vigilancia viral en aves y cerdos.
            \item Análisis geo-estadístico para el rastreo de enfermedades con Power BI y R.
            \item Implementación de Power BI en el departamento de marketing.
            \item Automatización con IA en el procesamiento de documentos.
            %\item Ejecución de pruebas serológicas (ELISA) y de biología molecular (RFLP y qRT-PCR).
            \item Gestión del suministro de reactivos y materiales en el laboratorio de serología.
        \end{itemize}
    }

\cvsubsection{Experiencia académica}

    \cvevent{GWAS para la caracterización de maíz Ancho nativo}
    {Instituto de Biotecnología de la Universidad Nacional Autónoma de México (IBt -- UNAM)}
    {}
    {2023}
    {2023}
    {%
        \vspace{-11pt}
        \begin{itemize}[leftmargin=*]
        \setlength{\itemsep}{0cm}
          %\item Análisis estadísticos en la respuesta fenotípica del maíz a sequía en R.
          \item Generación de un protcolo bionformático en HTML usando R Markdown.
          \item Manejo de bases de datos genómicos con R.
          %\item Manejo y análisis de bases de datos gubernamentales (agricultura).
          \item Análisis de Asociación Genómica (GWAS) con herramientas como GAPIT y TASSEL.
          \item Análisis de Ancestría con ADMIXTURE (Shell de Linux) y visualización con QGIS.
          \item Protocolo de secuenciación NGS DArTseq.
        \end{itemize}
    }
    
    \cvevent{Líder de proyecto - Competencia internacional (iGEM Design League)}
    {BerryVax (ITESO), iGEM Design League}
    {}
    {2021}
    {2022}
    {
        \vspace{-11pt}
        \begin{itemize}[leftmargin=*]
        \setlength{\itemsep}{0cm}
          \item Premiados con Medalla de Plata y Mejor Presentación de Proyecto.
          \item Manejo y análisis de bases de datos gubernamentales (agricultura).
          \item Proyecto a nivel Latino América.
          % \item Entrevistas con agricultores y empresarios para estrategias de mercado.
          % \item Capacitación del equipo técnico en Biología Sintética.
        \end{itemize}
    }
    
    % \cvevent{Participante - Competencia internacional (iGEM)}
    % {RubisCO (ITESO), iGEM}
    % {}
    % {2019}
    % {2020}
    % {
    %    \vspace{-11pt}
    %    \begin{itemize}[leftmargin=*]
    %    \setlength{\itemsep}{0cm}
    %     \item Proyecto a nivel mundial, presentación en Boston, MA.
    %     % \item Manejo y análisis de bases de datos gubernamentales en materia de saneamiento de agua tratada.
    %     % \item Entrevistas para estrategias de mercado y validación de proyecto. 
    %    \end{itemize}
    % }

% ------------
% EDUCATION
% ------------
\vspace{0.5em}
\cvsection{\education Educación}
    \cvevent{Licenciatura en Ingeniería en Biotecnología} % Carrera/posgrado/puesto
    {Instituto Tecnológico y de Estudios Superiores de Occidente (ITESO)} % Institución/empresa
    {} % Lugar
    {2018} % Inicio
    {2023} % Fin
    {
        \textbf{Tesis:} {Estudio de Asociación de Genoma Completo (GWAS) para la identificación de genes relacionados a la respuesta hidrotrópica en maíz ancho (\textit{Zea mays} L.) nativo mexicano}
    } % Texto
    
% ------------
% HABILIDADES
% ------------
\newpage
\setlength\columnsep{0.33in}
\begin{multicols}{2}
\cvsection{\large\lang Idiomas}
\vspace{-8pt}
\textbf{Inglés}\\
    \cvskill{Escritura}{5}
    \cvskill{Lectura}{5}
    \cvskill{Conversación}{5}

\medskip

\textbf{Alemán}\\
    \cvskill{Escritura}{2}
    \cvskill{Lectura}{3}
    \cvskill{Conversación}{1}

\bigskip

\cvsection{\large\courses Formación complementaria} % Cursos, talleres y certificaciones.
\vspace{-8pt}
{\small
  \begin{itemize}[leftmargin=*]
  \setlength{\itemsep}{6pt}
    \item \textbf{Curso de \say{Herramientas Bioinformáticas de Secuenciación Masiva}}\\julio 2023 (61 horas) en \textit{IBt - UNAM (USMBB)}. Fui asistente en el taller de \say{Programación con R}.
    \item \textbf{Clase complementaria: \say{Planeación de la Producción}}\\enero a mayo 2023 en \textit{ITESO}.
    \item \textbf{Clase complementaria: \say{Diseño de Información}}\\enero a mayo 2023 en \textit{ITESO}.
    \item \textbf{Clase complementaria:  \say{Métodos de Machine Learning}}\\enero a mayo 2022 en \textit{ITESO}
    \item \textbf{Clase complementaria: \say{Administración de Proyectos PMI}}\\agosto a diciembre 2021 en \textit{ITESO}.
    \item \textbf{Curso de: \say{Biología sintética y circuitos genéticos 101}}\\julio 2021 (28 horas) en \textit{Scintia}.
    \item \textbf{Taller: \say{Malice Analisys for your project}}\\mayo 2021 (12 horas) en \textit{Engineering Biology Research Consortium (EBRC)}.
    \item \textbf{Curso de: \say{Agile Process, Project, and Program Controls}}\\febrero 2021 (28h), curso asíncrono de \textit{HarvardX}.
    \item \textbf{Curso de \say{Data Science: R Basics}}\\agosto 2020 (60 horas), curso en línea de \textit{HarvardX}.
  \end{itemize}
}

\bigskip

\cvsection{\large\hardskill Hard skills}
\vspace{-6pt}
\textbf{Software y Herramientas}\\
    % \cvskill{Linux (Desktop)}{5}
    \cvskill{Archlinux (daily-driver)}{5}
    \cvskill{CentOS (server-driver)}{4}
    \cvskill{Git}{4}
    % \cvskill{Vim/NeoVim}{4}
    \cvskill{PowerBI}{4}
    \cvskill{Tableau}{4}
    \cvskill{Adobe Suite}{3}
    \cvskill{Amazon AWS}{1}
    \cvskill{Microsoft Azure}{1}
    % \cvskill{MBA3}{3}
    % \cvskill{SAP B1}{2}

\medskip

\textbf{Lenguajes de programación}\\
    \cvskill{R}{5}
    \cvskill{BASH}{5}
    \cvskill{SQL}{4}
    \cvskill{Python}{3}
    \cvskill{MATLAB}{3}
    \cvskill{VBA}{2}
    
\bigskip

\cvsection{\large\group Softskills} % Cursos, talleres y certificaciones.

\cvtag{Trabajo colaborativo}
\cvtag{Comunicación efectiva}
\cvtag{Oriendación humana/social}
\cvtag{Equipos multidisciplinarios}
\cvtag{Liderazgo}
\cvtag{Administración}
\cvtag{Diseño (UI) y creatividad}
\cvtag{Autodidacta}

% \cvsection{\publications Publicaciones}
% \defbibnote{mynote}{$^*$ Indica contribuciones iguales;\\$^\dagger$ Indica investigador(a) principal; y\\ \textbf{bold} Indica compañeros(as) de laboratorio.}

% \cvsubsection{Publicados (peer-reviewed)}

% \begin{spacing}{1.25}
%     \printbibliography[heading=none,keyword=peerreviewed,prenote=mynote]
% \end{spacing}

% \cvsubsection{En revisión (pre-prints)}

% \begin{spacing}{1.25}
%     \printbibliography[heading=none]
% \end{spacing}
\end{multicols}
\end{document}
